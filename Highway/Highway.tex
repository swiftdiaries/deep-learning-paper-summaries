\documentclass[a4paper]{article}

\usepackage{fullpage} % Package to use full page
\usepackage{parskip} % Package to tweak paragraph skipping
\usepackage{amssymb}
\usepackage{tikz} % Package for drawing
\usepackage{amsmath}
\usepackage{hyperref}
\usepackage{bm}

\title{Training Very Deep Networks}
\date{}

\begin{document}

\maketitle

\section{Citation}
Srivastava, Rupesh K., Klaus Greff, and Jürgen Schmidhuber. "Training very deep networks." Advances in neural information processing systems. 2015.

\begin{verbatim}
http://papers.nips.cc/paper/5850-training-very-deep-networks.pdf
\end{verbatim}

\section{Abstract}
To enable training deep networks, we introduce highway networks, which are
stacks of highway modules that have a gate that decides whether how much
of the layer input and layer output should be let through.

\section{Introduction and Previous Work}
Deeper networks have enabled us to improve accuracy on many machine learning
problems (e.g. ImageNet). To deal with the difficulty of training deep nets,
people have developed better optimizers, intialization strategies, skip
connections, distillation (i.e. teacher network teaches student network
with soft targets), and layerwise training. The key problem is that of vanishing
and exploding gradients. LSTMs mitigate this problem for recurrent neural nets
with their gating mechanism, so we apply the same principle here.

\section{Highway Networks}
A network layer computes $\bm{y} = H(\bm{x}, W_H)$ where $W_H$ is the weight
matrix and $H$ is usually an affine transformation (or convolution) followed
by a nonlinearity. For highway networks, we replace this with:

\begin{align}
  \bm{y} = H(\bm{x}, W_H) \cdot T(\bm{x}, W_T) + \bm{x} \cdot
  (1 - T(\bm{x}, W_T))
\end{align}

where $T$ is the transform gate that decides how much of the input and output
should be allowed through. To make $\bm{x}$ have the same dimension as $T$
and $H$, you can zero-pad it (if too short) or subsample it (if too long).
You can also apply $T$ convolutionally in the case of conv layers. We set
$T = \sigma(W_T^T \bm{x} + \bm{b}_T)$. We initialize the bias with negative
values so the sigmoid is near zero and thus the transform gate starts out by
allowing the input to flow through. We were able to train 1000 layer networks
usign this highway module described above.

\section{Experiments}
We train with SGD with momentum. The learning rate starts at $\gamma$ and decays
according to factor $\lambda$. We use ReLU activation.

We train on MNIST. We have thin networks with either 50 highway blocks or 71
plain blocks (to make sure highway and non-highway nets have about the same
number of parameters). Hyperparameters were chosen with random search. Our
highway networks converge faster than plain networks and outperform plain nets
with the same number of layers. We are competitive with state of the art with
many fewer parameters.

We also train on CIFAR-100, where we use Fitnets as our basis. These networks
use maxout units and are hard to train when they are deep. We add highway
blocks to these networks and are able to train them without problems. We also
make them deeper and can train those as well.

\section{Analysis}
We initialized transform gate biases with negative values, but we find that the
trained network actually makes them more negative (where deeper layer are even
more negative). At the lower layers, this is becuase the network is more
selective - producing transformations for specific kinds of samples and just
letting others flow through. For a single example, it won't trigger many blocks,
but the set of blocks triggered likely varies sample to sample. So, the
gating mechanism not only eases training, but also serves as a routing mechanism
that is essential for computation.

We try lesioning (i.e. set all the transform gates to zero so it just copies
the input) layers. For MNIST, the early layers are essential but the
layers after layer 10 have no impact on test error. This indicates that MNIST
is pretty easy and thus the network doesn't need most of its layers. For the
more difficult CIFAR-100, on the other hand, lesioning any layer but the last
few hurts test error, which indicates the network uses almost all of its layers.

\section{Discussion}
Good intialization, local response normalization, and deep supervision all
help train deeper models, but they struggle with very deep models. Highway
networks can train very deep models, allowing the network to learn which
layers to keep and which ones to ignore (i.e. close transform gate). They can
also help visualize how much computation depth is needed for a problem.

\end{document}
