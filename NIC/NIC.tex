\documentclass[a4paper]{article}

\usepackage{fullpage} % Package to use full page
\usepackage{parskip} % Package to tweak paragraph skipping
\usepackage{amssymb}
\usepackage{tikz} % Package for drawing
\usepackage{amsmath}
\usepackage{hyperref}

\title{Show and Tell: A Neural Image Caption Generator}
\date{}

\begin{document}

\maketitle

\section{Citation}
Vinyals, Oriol, et al. "Show and tell: A neural image caption generator." Proceedings of the IEEE conference on computer vision and pattern recognition. 2015.

\begin{verbatim}
https://www.cv-foundation.org/openaccess/content_cvpr_2015/papers/
Vinyals_Show_and_Tell_2015_CVPR_paper.pdf
\end{verbatim}

\section{Abstract}
Our Neural Image Captioning (NIC) model combines a CNN with an LSTM to create
descriptions of images. It gets state of the art BLEU scores
(ours/previous/human) on PASCAL (59/29/69), Flikr30k (66/56/68), SBU (28/19/?),
COCO (28/?/?).

\section{Introduction}
The goal is to take an image $I$ and create a sentence of words
$S = [S_1, S_2, ...]$ to describe the image.

In the world of translation, people solve this by using a Recurrent Neural
Network (RNN) for encoding the input sequence into a fixed size state and then
another RNN for decoding the state into a sequence. In our case, we get rid
of the encoder RNN and use a Convolutional Neural Network (CNN) to encode the
image into a fixed size vector.

\section{Related Work}
Some previous systems used And-Or graphs with custom rules. These were extremely
brittle and did not generalize.

Graph-based systems are slightly better, but they require a lot of
hand-engineered pieces.

Other work takes a predetermined set of descriptions and ranks them for a given
image. We don't think this approach will generalize well.

Some work uses a feed-forward neural network or a general RNN (not an LSTM like
we do). Recent work uses a two-stream approach with a CNN and LSTM, but they
focus on ranking.

\section{Model}
We parametrize our model by $\theta$. We seek:

$$
\theta^* = \textrm{arg max}_{\theta}{\sum_{(I, S)}{\log p(S | I, \theta)}}
$$

For a single sentence $S = [S_1, S_2, ..., S_N]$, the $p(S|I, \theta)$
decomposes into the following (we omit $\theta$ for brevity).

$$
\log p(S|I) = \sum_{t=0}^{N}{\log p(S_t|I, S_0, S_1, ..., S_{t-1})}
$$

We model the probability with a Long Short-Term Memory (LSTM) network, which
takes the following form:

$$
i_t = \sigma(W_{ix} x_t + W_{im} m_{t-1})
$$
$$
f_t = \sigma(W_{fx} x_t + W_{fm} m_{t-1})
$$
$$
o_t = \sigma(W_{ox} x_t + W_{om} m_{t-1})
$$
$$
c_t = f_t \odot c_{t-1} + i_t \odot \tanh(W_{cx} x_t + W_{cm} m_{t-1})
$$
$$
m_t = o_t \odot c_t
$$
$$
p_{t+1} = \textrm{Softmax}(m_t)
$$

where $i$, $f$, and $o$ are the input, forget, and output gates, respectively.
$\sigma$ is the sigmoid function. $c$ is the memory cell of the LSTM.

Now, we need to pick our $x_t$. We begin by setting $x_{-1} = \textrm{CNN}(I)$.
We use the GoogLeNet CNN. Then, we set $x_t = W_e S_t$ for all $t \in
\{0...N-1\}$, where $S_t$ is the one-hot encoded $t^{th}$ word in the ground
truth sentence and $W_e$ is an embedding matrix. Notice that our image
descriptor and word embedding vector must have the same size in order for
this to work.

Our loss function is then:

$$
L(I, S) = - \sum_{t=1}^{N}{\log p_t(S_t)}
$$

We backpropagate to optimize the LSTM parameters, $W_e$, and top layer of the
CNN.

At test time, we don't have a ground truth sentence, so how do we pick the
$x_t$ values for $t \ne -1$? One option is to sample from $p_1$ and and then
feed the LSTM's predictions back into itself. A better option is Beam Search,
where we keep track of the $k$ best sentences of length $t$ and use them
to generate the best $x_{t+1}$ (we keep the top $k$ of these too). We set
$k = 20$.

\section{Experiments}
We measure performance using BLEU-1 (or BLEU-4 for MS COCO). We also set up
an Amazon Mechanical Turk experiment to ask humans to rate our descriptions
as well. Another metric is perplexity, which we use for hyperparameter tuning,
but we don't actually use as a final metric.

We train on SBU (large but noisy), MS COCO (great dataset), Flickr30k (smaller),
Flickr8k (smaller) and evalate on held-out portions of each and on PASCAL.

We use a CNN pretrained on ImageNet and fine-tune the last layer.
We tried training an embedding matrix $W_e$
from a big news dataset, but it didn't help much so we just train it from
scratch along with the LSTM. The LSTM and word embedding has 512 dimensions.

We get a huge improvemnt over state-of-the-art for SBU and PASCAL because the
previous work did not use neural networks. We also set state of the art for the
other datasets, but we aren't at human level performance yet.

We tried doing transfer learning between Flick30k and Flickr8k and find that
the model transfers well and beats training from scratch on Flickr8k.
Going from Flickr30k to MS COCO is not great (it's better to train from
scratch), but the descriptions are still ok.

Looking at some of the top $k$ beam search outputs, we see that captions look
pretty good (and diverse too).

We also compare our system to other ranking based approaches and we do
reasonably well on the ranking metrics as well.

The human evaluation of our model shows that it is worse than the human
captions, more so than the BLEU scores would suggest. This indicates that
BLEU is not a perfect alternative to human evaluation.

Looking at our word embeddings, we find that nearest words make sense. For
example, "car" is near "van" and "cab".

\section{Conclusion}
NIC is an end-to-end neural network for image captioning. It combines a CNN
with an LSTM to maximize the probability of a sentence. It does well on BLEU
and human evaluation. Future work should see how unsupervised learning can
improve the system.




\end{document}
