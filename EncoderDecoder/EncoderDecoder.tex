\documentclass[a4paper]{article}

\usepackage{fullpage} % Package to use full page
\usepackage{parskip} % Package to tweak paragraph skipping
\usepackage{amssymb}
\usepackage{tikz} % Package for drawing
\usepackage{amsmath}
\usepackage{hyperref}

\title{Learning Phrase Representations using RNN Encoder–Decoder for
Statistical Machine Translation}
\date{}

\begin{document}

\maketitle

\section{Citation}
Cho, Kyunghyun, et al. "Learning phrase representations using RNN encoder-decoder for statistical machine translation." arXiv preprint arXiv:1406.1078 (2014).

\begin{verbatim}
https://arxiv.org/pdf/1406.1078.pdf
\end{verbatim}

\section{Abstract}
In translation, you must turn an input sequence in one language into an
output sequence in another language. To do this, we've created a neural network
where an Encoder Recurrent Neural Network (RNN) turns the input sequence into
a fixed length context and then we feed that context into a Decoder RNN that
produces the output sequence.

\section{Introduction}
Statistical Machine Translation (SMT) is a system that can translate from one
language into another (we consider English to French). Part of the system
involves scoring phrase pairs. We use an Encoder-Decoder neural network to do
the scoring. We also have a sophisticated hidden unit (based on LSTM) that we
use in our network. The context learned by the Encoder-Decoder preserves
semantic and syntactic structure of the phrase and leads to improved SMT
performance.

\section{RNN Encoder Decoder}
Given input words $\mathbf{x} = [x_1, x_2, ..., x_{T}]$, we compute hidden state

$$
\mathbf{h}_t = f(\mathbf{h}_{t-1}, x_t)
$$

We then set $\mathbf{c} = h_T$ as our context. Now, the decoder network produces
and output sequence $\mathbf{y} = [y_1, ..., y_{T'}]$ where:

$$
\mathbf{h}_t = f(\mathbf{h}_{t-1}, y_{t-1}, \mathbf{c})
$$

$$
p(y_t | y_{t-1}, ..., y_1, \mathbf{c}) = g(\mathbf{h}_t, y_{t-1}, \mathbf{c})
$$

where $\mathbf{h}$ and $f$ are different than in the encoder RNN.

We can then train this network end-to-end to maximize log likelihood:

$$
\max_{\theta}{\frac{1}{N} \sum_{n=1}^{N}{\log{p_{\theta}(
\mathbf{y}_n, \mathbf{x}_n
)}}}
$$

For $f$ in the equations above, we use the following hidden unit, which is based
on the Long Short-Term Memory (LSTM) unit. We start with a reset gate:

$$
r_j = \sigma([\mathbf{W}_r \mathbf{x}]_j + [\mathbf{U}_r \mathbf{h}_{t-1}]_j)
$$

Then we compute an update gate:

$$
z_j = \sigma([\mathbf{W}_z \mathbf{x}]_j + [\mathbf{U}_z \mathbf{h}_{t-1}]_j)
$$

Now we compute the activation of the hidden unit:

$$
\tilde{h}_j^t = \phi([\mathbf{W} \mathbf{x}]_j + [\mathbf{U}(
\mathbf{r} \odot \mathbf{h}_{t-1}
)]_j)
$$

$$
h_j^t = z_j h_j^{t-1} + (1 - z_j) \tilde{h}_j^{t}
$$

We also tried a $\tanh$ unit, but did not get gains.

\section{Statistical Machine Translation}
An SMT system figures out the most probable translation $\mathbf{f}$ of an input
sequence $\mathbf{e}$ using:

$$
p(\mathbf{f} | \mathbf{e}) = \sum_{n=1}^{N}{w_n f_n(\mathbf{f}, \mathbf{e})
+ \log{Z(e)}}
$$

where $w_n$ denotes weights and $f_n$ denotes features. Our metric is BLEU
score. In the SMT system, we need to assign scores to phrase pairs. We use
our neural network for this purpose.

When training, we sample from a uniform distribution over phrase pairs. We
don't sample by phrase frequency to avoid training the network to just remember
the most common phrases. In theory, you can replace the generation of
the phrase table with our neural network, but we don't do that in this
paper.

Other papers have used feedforward neural nets to score phrases, but they
require fixed size phrases. One paper embeds words in both languages into
the same space and then maps one phrase to another by finding the closest
phrase in the other language. There are other RNN-based approaches.

\section{Experiments}
We do English to French translation. We use multiple datasets, but use a
data selection technique to pick a 418M word corpus from a 2B one. Historically,
training on the entire corpus doesn't give best performance. Our vocabulary has
15K words and the rest are treated as "unknown" symbols.

Our neural net has 1000 hidden units and matrix multiplication is approximated
with two low-rank (100) matrices instead of one big matrix. The activation
function in $\tilde{h}$ is $\tanh$. The output of the decoder goes through
a layer with 500 maxout units (each pooling 2 inputs). Non-recurrent weights
are sampled from Gaussians, but recurrent weights come from a white Gaussian's
left singular vectors matrix. We train with Adadelta with a batch size of 64
for three days.

We also tried scoring phrase pairs with a CLSM language model. We find that
we get best performance when we ensemble the CLSM and our neural network,
indicating that they are complementary. If you must pick one, our neural
network is better.

Looking at some phrase pairs, our neural network tends to go for literal
translations, prefers shorter phrases, and doesn't just memorize the most
frequent phrases. When you sample from the network, you get phrases that
are not in the phrase table, indicating this might a good way to generate
the phrase table as well.

If we look at the phrases in the learned embedding space, we find that our
neural network learns semantic and syntactic similarities.

\section{Conclusion}
Our Encoder-Decoder structure turns one variable length sequence into another.
We also have a new hidden unit, based on the LSTM, to use with this
network. Using our network to score phrases increases BLEU score. It also
seems to generate good phrases, so maybe you can use it to generate the phrase
table.



\end{document}
