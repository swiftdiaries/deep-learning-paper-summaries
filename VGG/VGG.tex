\documentclass[a4paper]{article}

\usepackage{fullpage} % Package to use full page
\usepackage{parskip} % Package to tweak paragraph skipping
\usepackage{tikz} % Package for drawing
\usepackage{amsmath}
\usepackage{hyperref}

\title{Very Deep Convolutional Networks For Large Scale Image Recognition}
\date{}

\begin{document}

\maketitle

\section{Citation}

Simonyan, Karen, and Andrew Zisserman. "Very deep convolutional networks for large-scale image recognition." arXiv preprint arXiv:1409.1556 (2014).

https://arxiv.org/pdf/1409.1556.pdf

TODO: Put citation here.

\section{Abstract}
We won ImageNet 2014 Classification + Localization with a deep (16-19 layer)
CNN with small (3 $\times$ 3) filters.

\section{Introduction}
Ever since the AlexNet CNN work ImageNet 2012, people have been improving CNNs.
We build a very deep CNN (enabled by having filters with small receptive fields)
that gets state of the art performance on ImageNet classification and
localization.

\section{ConvNet Configurations}
Input is $224 \times 224$ image where mean RGB value is subtracted from each
pixel. Each of our CNNs uses $3 \times 3$ filters (we pick this so we can look
at the upper, lower, left, and right pixel). Some use $1 \times 1$ filters,
which is just a linear transformation of input channels. We use $2 \times 2$
spatial pooling with a stride of 2. Our CNNs end with three fully connected
layers and a softmax - the first and second FCs have 4096 units and the last FC
has 1000 units. We use ReLU activation. We tried Local Response Normalization
(like in AlexNet), but it didn't help.

Our smallest CNN (called network A) has 11 layers and the largest (network E)
has 19.

Notice that three stacked $3 \times 3$ filters has an effective receptive field
of $7 \times 7$, but only uses $3(3 \times 3)C^2 = 27C^2$ parameters as opposed
to the $(7 \times 7)C^2 = 49C^2$ parameters of a $7 \times 7$ filter. The three
small filters have a regularizing effect. The $1 \times 1$ filters (+ ReLU)
add more nonlinearity to our decision function.

\section{Classification Framework}
We optimize softmax regression loss using minibatch ($B = 256$) gradient descent
with momentum. We regularizing with L2 weight decay and dropout. Learning rate
is cut by a factor of 10 each time validation error stops improving. We train
for 74 epochs which is less than AlexNet because we have greater regularization
and we pre-initialize certain layers. We trained network A with random
initialization. Then, for the deeper networks, we initialize the first 4 layers
and FC layers with the weights with network A.

We randomly sample crops from different scales, do horizontal flips, and do RGB
color shift to augment the dataset.

When taking a crop, we first scale the image so that shortest size has length
$S$. How do you pick $S$? We tried fixing it at $S = 256$ and $S = 384$, and we
also tried sampling it uniformly from $S \in [S_{min}, S_{max}] = [256, 512]$.
For this multiscale model, we just fine-tuned the $S = 384$ model.

At test-time, we scale the image so the shortest side has length $Q$. We apply
our CNN densely (also tried multi-scale/crops) over the image and sum-pool
the resulting spatial map. We do the same thing for the flipped image and take
the average between the flipped and non-flipped images as our prediction.

We built our CNNs using a modified Caffe library and takes 2-3 weeks to train on
4 NVIDIA Titan GPUs.

\section{Classification Experiments}
ImageNet classification has 1.3M/50K/100K training/validation/test images.

We set $Q = S$ for fixed-$S$ and $Q = 0.5(S_{min} + S_{max})$ for jittered $S$.

Local Response Normalization does not help.

Deep network with small filters $>$ Shallow network with large filters.

Scale jittering $>$ single scale. Scale jittering at test time also helps.

At test time, multi-crop is slightly better than dense application of the CNN,
and combining them both is the best.

Averaging all the CNNs we made does well. Averaging the two best CNNs does even
better.

We did not win ILSVRC classification (GoogLeNet did), but we won localization.

\section{Conclusion}
Deep networks with small filters work well for classification and other tasks.

\section{Appendix A - Localization}
We use network D and modify it to predict a bounding box (a 4D vector with the
center coordinate, height, and width). We actually found that having a
per-class bounding box (so the vector would be $1000 \times 4 = 4000D$), did
better than having one bounding box for all classes.

We use fixed-$S$ and train on Euclidean loss.

We considered two testing protocols - apply to central crop and densely apply
network. For the latter, use the greedy bounding-box merging strategy from
OverFeat. The latter worked better.

In ImageNet, a localization is correct if the intersection over union of the
bounding boxes is at least 50\%.

\section{Generalization of Very Deep Features}
We remove the last fully connected layer for network D and network E - this
gives a feature extractor that you can use for other tasks. All you need to
do scale your image to size $Q$, densely apply the CNN to get feature vectors,
L2 normalize the feature descriptors, global average pool them (or stack them),
and feed into an SVM.

We do pretty well on PASCAL VOC and Caltech datasets.

\end{document}
