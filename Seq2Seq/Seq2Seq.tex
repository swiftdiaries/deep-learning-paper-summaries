\documentclass[a4paper]{article}

\usepackage{fullpage} % Package to use full page
\usepackage{parskip} % Package to tweak paragraph skipping
\usepackage{amssymb}
\usepackage{tikz} % Package for drawing
\usepackage{amsmath}
\usepackage{hyperref}

\title{A Convolutional Neural Network for Modelling Sentences}
\date{}

\begin{document}

\maketitle

\section{Citation}
Sutskever, Ilya, Oriol Vinyals, and Quoc V. Le. "Sequence to sequence learning with neural networks." Advances in neural information processing systems. 2014.

\begin{verbatim}
http://papers.nips.cc/paper/5346-sequence-to-sequence-learning-with-neural-networks.pdf
\end{verbatim}

\section{Abstract}
We translate from English to French with an encoder-decoder network. Our encoder
is a stack of LSTMs that produces a fixed size output. Our decoder takes
the this as input and produces a variable length output. A key insight we
make is that the input sentence should be fed in reverse order. We get a BLEU
score of 34.8 compared to 33.3 from a classic phrase based Statistical
Machine Translation (SMT) system. Using our network to rank the SMT phrase
pairs gets a BLEU score of 36.5 (near state of the art).

\section{Introduction}
A key limitation of neural networks is they require a fixed-size input. An
encoder-decoder network overcomes this. We focus on the WMT'14 English to
French translation task. We penalize our system when it encounters words
outside its vocabulary. Reversing input sentences introduces short term
dependencies to improve the model.

\section{The Model}
We stack LSTMs to map the reversed input sentence into a fixed size vector.
We then use another stack of LSTMs to map the fixed sized vector into an
output sequence. Sentences are terminated with an end-of-sentence (EOS) word.
We use four-layer LSTMS. The final LSTM feeds into a softmax over the
vocabulary.

To understand why reversing the input sentence is important, suppose that
we want to map the sequence $a, b, c$ to $\alpha, \beta, \gamma$, Notice that
in the sequence $a, b, c, \alpha, \beta, \gamma$ that $a$ and $\alpha$ are
very far apart. This means the output LSTMs will need a good memory to remember
what started the sentence. On the other hand, if we feed in a reversed
sentence, then the sequence $c, b, a, \alpha, \beta, \gamma$ has $a$ and
$\alpha$ right near each other and also has $b$ and $\beta$ reasonably
close as well. This encourages better gradient flow.

\section{Experiments}
We train on 12M sentences (348M French words, 304M English words). We use
160K frequent words from the source language and 80K frequent words from
the target language. Our training objective is, for training set $D$,

$$
\frac{1}{|D|} \sum_{(T, S) \in D}{\log{p(T|S)}}
$$

At test time, we seek $\hat{T} = \textrm{argmax}_{T}{p(T|S)}$. We find this
with beam search (beam size 2).

Reversing the input sentence boosts the BLEU score by about 4.

LSTMs have 1000 cells. Word embeddings have size 1000. LSTM parameters are
initialized from a uniform distribution between $[-0.08, 0.08]$. We train
for 7.5 epochs using SGD (learning rate 0.7) with momentum. After 5 epochs,
we halved the learning rate every 0.5 epochs. Our batch size is 128. We do
gradient clipping by computing $s = ||g||$ where $g$ is the gradient
divided by 128. If $s > 5$, we set the gradient to $\frac{5g}{s}$. We made
sure sentences in a minibatch were about the same length to avoid making
short sentences wait for long sentences to finish processing.

See the abstract of this summary for the metrics. The LSTM does well even
if the sentence is long. We project our fixed-size vectors into 2D space
with PCA and find that the vector captures word order and insensitive to
passive vs. active voice.

\section{Related Work}
Most neural network translation work so far just rescores the phrase pairs of
an SMT. Some people use Convolutional Neural Networks (CNNs) to turn sentences
into vectors, but this loses word order. Others use attention mechanisms.

\section{Conclusion}
Our encoder-decoder LSTM-based network beats the SMT. We reverse the
source sentence for big gains. We get good performance on long sentences while
other researchers previous work struggled with them.




\end{document}
